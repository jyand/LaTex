\documentclass[12pt]{article}
\usepackage{amsmath}
\usepackage{amssymb}
\def\upint{\mathchoice%
    {\mkern13mu\overline{\vphantom{\intop}\mkern7mu}\mkern-20mu}%
    {\mkern7mu\overline{\vphantom{\intop}\mkern7mu}\mkern-14mu}%
    {\mkern7mu\overline{\vphantom{\intop}\mkern7mu}\mkern-14mu}%
    {\mkern7mu\overline{\vphantom{\intop}\mkern7mu}\mkern-14mu}%
  \int}
\def\lowint{\mkern3mu\underline{\vphantom{\intop}\mkern7mu}\mkern-10mu\int}
\renewcommand{\baselinestretch}{1.5}

\begin{document}
\pagestyle{plain}

\title{Real Analysis: Midterm 2 Review}
\author{John DeSalvo}
\maketitle

\tableofcontents

\section{L'Hospitals Rule}


\section{Taylor's Theorem}


\section{Integral Defined}
a Partition $P = \{x_{0}, x_{1}, ..., x_{n}\}$ of $[a,b]$ is a set of subintervals $[x_{0}, x_{1}], [x_{1}, x_{2}], ..., [x_{n-1}, x_{n}]$ where $a = x_{0} < x_{1} < ... <x_{n}$ \\
the Norm of a Parition $||P|| = max_{1 \le i \le n}(x_{i} - x_{i-1})$ is the largest of the lengths of all subintervals \\
\\
$P'$ is a Refinement of $P$ if both $P$ and $P'$ are partitions of $[a,b]$ and every parition point of $P$ is also a partition point of $P'$ (i.e. $P'$ has additional points between those of $P$) \\
\\
the Reimann Sum of $f$ (provided $f$ is defined on $[a,b]$) over $P$ is $\sigma = \sum_{j=1}^{n}f(c_{j})(x_{j} - x_{j-1})$ where $x_{j-1} \le c_{j} \le x_{j}$ , $1 \le j \le n$   There are infinitely many Reimann Sums over a given Partition for a given function \\
\\
$f$ is Reimann Integrable on $[a,b]$ if $\exists L$ s.t.:  $\forall \epsilon > 0$ $\exists \delta > 0$ s.t. $|\sigma - L| < \epsilon$ where the $\sigma$ satisifes $||P|| < \delta$ , i.e. the Reimann Integral is $\int_{a}^{b}f(x)dx = L$ , which if it exists, is unique \\
\\
Theorem: if $f$ is unbounded on $[a,b]$ , then $f$ is not integrable on $[a,b]$ (proof on p. 119 in Trench)\\
\\
for $f$ bounded on $[a,b]$ and $P$ on $[a,b]$ , let $M_{j} = sup_{x_{j-1} \le x \le x_{j}}(f(x))$ and $m_{j} = inf_{x_{j-1} \le x \le x_{j}}(f(x))$ , the Upper Sum of $f$ over $P$ is $S(P) = \sum_{j=1}^{n}M_{j}(x_{j-1} - x_{j}$ and the Upper Integral is
\begin{gather*}
   \upint_a^b f(x)\,\mathrm{d}x 
\end{gather*} 
is the infimum of all upper sums, the Lower Sum of $f$ over $P$ is $s(P) = \sum_{j=1}^{n}m_{j}(x_{j-1} - x_{j}$ and the Lower Integral is 
\begin{gather*}
   \lowint_a^b f(x)\,\mathrm{d}x
\end{gather*} 
is the supremeum of all lower sums\\
Theorem: Let $f$ be bounded on $[a,b]$ and $P$ be a partition of $[a,b]$ . $s(P)$/$S(P)$ is the supremum/infimum of the set of all Reimann sum of $f$ over $P$ (proof on p. 121 in Trench)\\
\\
The Riemann-Stieltjes Integral: Let $f$ and $g$ on $[a,b]$ if $\exists L$ s.t. $\forall \epsilon > 0$ $\exists \delta > 0$ s.t. $|\sum_{j=1}^{n}f(c_{j})[g(x_{j}) - g(x_{j-1})] - L| < \epsilon$ , provided $||P|| < \delta$, then $f$ is Riemann integrable on $[a,b]$ so \\
$\int_{a}^{b}f(x)dg(x) = L$ is the Riemann-Stieltjes integral

\section{Integral Existence}
\textbf{*note: be able to prove these theorems}
Lemma: suppose $|f(x)| \le M$ , $a \le x \le b$ and let $P'$ be a partition of $[a,b]$ obtained by adding $r$ points to a parition $P$ of $[a,b]$ \\
$S(P) \ge S(P') \ge S(P) - 2Mr||P||$ and \\
$s(P) \ge s(P') \ge s(P) - 2Mr||P||$ \\
\\
Theorem: if $f$ is bounded on $[a,b]$ then 
\begin{gather*}
   \lowint_a^b f(x)\,\mathrm{d}x \le \upint_a^b f(x)\,\mathrm{d}x \\
\end{gather*}
\\
Theorem: if $f$ is integrable on $[a,b]$ then
\begin{gather*}
	\lowint_a^b f(x)\,\mathrm{d}x = \upint_a^b f(x)\,\mathrm{d}x = \int_{a}^{b}f(x)dx \\
\end{gather*}
\\
Lemma: if $f$ is bounded on $[a,b]$ and $\epsilon > 0$ then \\
$\exists \delta > 0$ s.t.
\begin{gather*}
	\lowint_a^b f(x)\,\mathrm{d}x \le S(P) < \upint_a^b f(x)\,\mathrm{d}x + \epsilon \\
\end{gather*}
and
\begin{gather*}
	\lowint_a^b f(x)\,\mathrm{d}x \ge s(P) > \upint_a^b f(x)\,\mathrm{d}x - \epsilon \\
\end{gather*}
provided $||P|| < \delta$ \\
\\
Theorem: if $f$ is bounded on $[a,b]$ and
\begin{gather*}
   \lowint_a^b f(x)\,\mathrm{d}x = \upint_a^b f(x)\,\mathrm{d}x = L \\
\end{gather*} 
then $f$ is integrable on $[a, b]$ and $\int_{a}^{b}f(x)dx = L$

\section{Integral Properties}
First Mean Value Theorem for Integrals: Suppose $u$ is continuous and $v$ is integrable and non-negative on $[a,b]$ \\
then $\int_{a}^{b}u(x)v(x)dx = u(c) \int_{a}^{b}v(x)dx$ for some $c \in [a,b]$ \\
\\
Theorem: if $f$ is integrable on $[a,b]$ and $a \le a_{1} < b_{1} \le b$ then $f$ is integrable on $[a_{1},b_{1}]$ \\
\\
Theorem: suppose that $F$ is continuous on $[a,b]$ and differentiable on $(a,b)$ , $f$ is integrable on $[a,b]$ , and that $F'(x) = f(x)$  $a<x<b$ \\
$\int_{a}^{b}f(x)dx = F(b) - F(a)$ \\
Fundamental Theorem of Calculus: if $f$ is continuous on $[a,b]$ then $f$ has a derivative on $[a,b]$ \\
if $F$ is an antiderivative of $f$ on $[a,b]$ then $\int_{a}^{b}f(x)dx = F(b) - F(a)$ \\
\\
Theorem for Integration by Parts: if $u'$ and $v'$ are integrable on $[a,b]$ then $\int_{a}^{b}u(x)v'(x)dx = u(b)v(b) - u(a)v(a) - \int_{a}^{b}v(x)u'(x)dx$ \\
\\
Second Mean Value Theorem for Integrals: suppose $f$ is non-negative and integrable and $g$ is continuous on $[a,b]$ \\
$\int_{a}^{b}f(x)g(x)dx = f(a) \int_{a}^{c}g(x)dx f(b) \int_{c}^{b}g(x)dx$ for some $c \in [a,b]$ \\

\section{Improper Intergals}
\subsection{Local Integrability}
if $f$ is locally integrable on $[a,b)$ if the limit exists $\int_{a}^{b}f(x)dx = lim_{c \to b^{-}} \int_{a}^{c}f(x)dx$ , $\infty^{-}$ is equivalent to $\infty$ \\
if $f$ is locally integrable on $(a,b]$ if the limit exists $\int_{a}^{b}f(x)dx = lim_{c \to a^{+}} \int_{c}^{b}f(x)dx$ , $\infty^{+}$ is equivalent to $\infty$ \\
if $f$ is locally integrable on $(a,b)$ if the limit exists $\int_{a}^{b}f(x)dx = \int_{a}^{\alpha}f(x)dx + \int_{\alpha}^{b}f(x)dx$
\subsubsection{Theorem}
suppose $f_{1}, ..., f_{n}$ are locally integrable on $[a,b)$ and that $\int_{a}^{b}f_{1}(x)dx, ..., \int_{a}^{b}f_{n}(x)dx$ converge and let $c_{1}, ..., c_{n}$ be constants \\
$\int_{a}^{b}(c_{1}f_{1} + ... + c_{n}f_{n})(x)dx$ converges and $= c_{1} \int_{a}^{b}f_{1}(x)dx + ... + c_{n} \int_{a}^{b}f_{n}(x)dx$
\subsection{Improper Integrals of Non-negative Functions}
\subsubsection{Theorem}
\subsubsection{Theorem: Comparison Test}
\subsubsection{Theorem}
\subsection{Absolute Integrability}

\subsubsection{Theorem}
\subsection{Conditional Convergence}

\subsubsection{Theorem: Dirlecht's Test}
\subsection{Theorem}
\subsection{Theorem}

\section{Infinite Sequences}


\end{document}
