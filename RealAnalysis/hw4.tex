\documentclass[12pt]{article}
\usepackage{amsmath}
\usepackage{amssymb}
\begin{document}
\pagestyle{plain}

\title{Math 545: Real Analysis}
\author{John DeSalvo}
\maketitle

\section{44.c}
if $lim_{x \to x_{0}}f(x) = \infty$ then $lim_{x \to x_{0}}|f(x)|^{\frac{1}{f(x)}}$ has the $\infty ^{0}$ indeterminate form \\
$|f(x)|^{\frac{1}{f(x)}} = e^{\frac{ln|f(x)|}{f(x)}}$ \\
let $\frac{g(x)}{h(x)} = \frac{ln|f(x)|}{f(x)}$, $lim_{x \to \infty}\frac{g^{(n)}(x)}{h^{(n)}(x)} = 0$ since $lim_{x \to x_{0}}f(x) = \infty$, $g(x)$ and $h(x)$ can be differentiated until the result is $\frac{0}{1} = 0$ because $g(x) = ln|f(x)|$ and $h(x) = f(x)$ and since $f(x)$ has a limit that exists and $\ne 0$, there exists an nth derivative that is constant \\
so $lim_{x \to x_{0}}|f(x)|^{\frac{1}{f(x)}} = lim_{x \to \infty}|f(x)|^{\frac{1}{f(x)}} = lim_{x \to \infty}e^{\frac{ln|f(x)|}{f(x)}} = e^{0} = 1$

\section{16.d}
$f(x) = ln(x) \approx (x-1) - \frac{(x-1)^{2}}{2} + \frac{(x-3)^{3}}{3}$ with $n=3$ \\
via Taylor's Theorem, $R_{4}(x) = -\frac{(x-1)^{4}}{4}$ \\
$|x-1| < \frac{1}{64}$ so $ -\frac{1}{63} < x < \frac{5}{64}$ \\
since $R_{4}(x)$ is negative, the upper bound of the magnitude is $R_{4}(c)$ with $c = -\frac{1}{63}$ \\
$M_{4} = \frac{1}{4*63^{4}}$

\end{document}
